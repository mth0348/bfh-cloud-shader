\section{Planning}

\subsection{Schedule}
The time frame of the semester spans over exactly 16 weeks. Being worth 4 ECTS points, this project assumes a maximum work load of 8 hours per week, resulting in a total of 128 hours. 
\vspace{\baselineskip}
\\
Over the course of the term, the project will be split into four primary task groups, namely organization, research, prototyping and finalization.
Put into relation with the duration of the project, an estimated schedule looks like this:
\vspace{\baselineskip}

\begin{ganttchart}[
    vgrid={dotted},
    hgrid={draw=black!50, dotted},
    bar/.append style={fill=lightgray},
    x unit=0.65cm,
    milestone node/.append style={fill=orange}
    ]{1}{16}
    \gantttitle{Work weeks}{16} \\
    \gantttitlelist{1,...,16}{1} \\
    \ganttbar{Organization}{1}{4} \\
    \ganttmilestone{Specification finished}{4} \\
    \ganttbar{Research}{5}{11} \\
    \ganttbar{Prototyping}{7}{15} \\
    \ganttmilestone{Prototype 1 finished}{9} \\
    \ganttmilestone{Prototype 2 finished}{11} \\
    \ganttmilestone{Prototype 3 finished}{15} \\
    \ganttbar{Finalizing}{16}{16}
\end{ganttchart}

\vspace{\baselineskip}
\begin{flushleft}
The milestones \emph{Prototype finished} for prototypes 1, 2, and 3 are referring to the prototypes about volumetric rendering, noise generation and ray marching, respectively.
\\
\vspace{\baselineskip}
For each task group, the following distribution of time and effort is estimated:
\newline
\newline
\begin{tabular}{|c|c|}
    \hline
    Task group & predicted effort \\ \hline
    Organization & 10\% \\ \hline
    Research & 35\% \\ \hline
    Prototyping & 50\% \\ \hline
    Finalizing & 5\% \\ \hline
\end{tabular}
\end{flushleft} 