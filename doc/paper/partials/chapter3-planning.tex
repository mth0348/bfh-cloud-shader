\section{Planning}

\subsection{Schedule}
The time frame of the semester spans over exactly 16 weeks. Being worth 4 ECTS points, this project assumes a maximum work load of 8 hours per week, resulting in a total of 128 hours. 
\vspace{\baselineskip}
\\
Over the course of the term, the project will be split into four primary task groups, namely organization, research, prototyping and finalization.
Put into relation with the duration of the project, an estimated schedule looks like this:
\vspace{\baselineskip}

\begin{ganttchart}[
    vgrid={draw=none, dotted},
    bar/.append style={fill=lightgray},
    x unit=0.65cm
    ]{1}{16}
    \gantttitle{Work weeks}{16} \\
    \gantttitlelist{1,...,16}{1} \\
    \ganttbar{Organization}{1}{3} \\
    \ganttbar{Research}{2}{6} \\
    \ganttbar{Prototyping}{7}{14} \\
    \ganttbar{Finalizing}{15}{16}
\end{ganttchart}

\vspace{\baselineskip}
\begin{flushleft}
For each task group, the following distribution of time and effort is estimated:
\newline
\newline
\begin{tabular}{|c|c|}
    \hline
    Task group & predicted effort \\ \hline
    Organization & 10\% \\ \hline
    Research & 10\% \\ \hline
    Prototyping & 10\% \\ \hline
    Finalizing & 10\% \\ \hline
\end{tabular}
\end{flushleft}