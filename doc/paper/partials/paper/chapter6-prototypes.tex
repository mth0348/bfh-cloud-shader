\section{Prototypes and Results}

\subsection{Disclaimer}
All of the following documented procedures and algorithms were prototyped and implemented in 3D, but for the matter of explanation, it is described and visualized in 2D.

\subsection{First Draft}
The first drafts of prototypes created during this project all revolve around volumetric rendering. 
Instead of using a \gls{sdf}, a noise function was used. The primary issue was to get the cube transparent where the noise function would return a number close to 0.0 and to color it where the number would be close to 1.0.
The approach for solving this issue is done by sampling the cloud's density instead.

\subsubsection{Density sampling}
Like in \gls{volumetricrendering}, for each pixel fragment, a ray is cast from the fragment into the cube, along the view direction for that fragment.
Usually, the algorithm can stop for a given ray if the \gls{sdf} returns a small enough distance, meaning the ray has hit a surface of the volume. However, it is different in the case with clouds, where the volume is \textit{\gls{translucent}} at most points.
\\
To account for that, the ray does not stop until the end of the container cube is reached. It samples the density $N$ times along its path and returns the sum of those samples, giving an approximate qualifier for how dark this fragment should be.

\begin{figure}[H]
    \centering
    \begin{tikzpicture}[scale=1.2]
        \tikzset{edge/.style = {-{Latex[length=3mm]},shorten >= -4pt}}
        \tikzset{shortedge/.style = {-{Latex[length=3mm]},shorten <=-4pt,shorten >= -4pt}}
        \tikzset{icon/.style = {font=\Large}}

        % icons
        \node[icon,rotate=35,anchor=west] (cam) at (0, 0) {\faVideoCamera};

        % clouds
        \node[cloud, cloud puffs=15.7, minimum width=5cm, minimum height=3.5cm, align=center, draw] (cloud) at (4.5, 3.5) {};
        \node[cloud, cloud puffs=15.7, minimum width=2cm, minimum height=1.5cm, align=center, draw] (cloud) at (5.0, 1.1) {};
        \node[cloud, cloud puffs=15.7, minimum width=3cm, minimum height=2.0cm, align=center, draw] (cloud) at (0.2, 3.5) {};
        \node[cloud, cloud puffs=15.7, minimum width=1.5cm,minimum height= 1cm, align=center, draw] (cloud) at (1.75, 0.9) {};

        % outer point
        \node (pOut) at (7, 5.25) {};
        \draw[red, edge] (cam) -- (pOut) node[midway,above,sloped,xshift=-3cm] {};

        % cloud points
        \node[red] (p1) at (1.5, 1.125) {\textbullet};
        \node[red] (p2) at (2.5, 1.875) {\textbullet};
        \node[red] (p3) at (3.5, 2.625) {\textbullet};
        \node[red] (p4) at (4.5, 3.375) {\textbullet};
        \node[red] (p5) at (5.5, 4.125) {\textbullet};

        \node[red, yshift=0.4cm] at (p1) {$p_1$};
        \node[red, yshift=0.4cm] at (p2) {$p_2$};
        \node[red, yshift=0.4cm] at (p3) {$p_3$};
        \node[red, yshift=0.4cm] at (p4) {$p_4$};
        \node[red, yshift=0.4cm] at (p5) {$p_5$};


        \end{tikzpicture}
    \caption{Prototype: Density sampler ray with $N = 5$.}
    \label{img:tikz:prototypes:densitysampling}
\end{figure}

\noindent
Understandably, the bigger clouds in \autoref{img:tikz:prototypes:densitysampling} represent higher return values of the noise function, meaning denser areas.
For the displayed ray, the values for points $p_1, p_3, p_4$ and $p_5$ are accumulated and used as a qualifier to color the fragment. In this case, a rather dark tone would be used.
\\
It is notable that $N$ has a linear impact on the performance, so it should be chosen carefully.

\begin{figure}[H]
    \centering
    \includegraphics[width=\linewidth]{unity captures/prototype1.PNG}
    \captionof{figure}{Prototype: Rendered image of sampled density based on 3D Perlin noise.}
    \label{img:captures:prototype1}
\end{figure}

\noindent
With this first try, a Perlin noise function was sampled. The returned value had to be normalized in a range of $[0, 1]$ in order to for it to be used as alpha value of the color.

\subsubsection{Normalizing Density}
This is where the exponential function $exp(x) = e^{-x}$ comes in, which (when clamped from 0 to 1) converts very low values to 1.0 and higher values will converge towards 0.0.

\begin{figure}[H]
    \centering
    \begin{minipage}{0.47\linewidth}
        \begin{tikzpicture}
            \begin{axis}[
                axis lines=center,
                samples=50,
                xmin=-0.5,
                xmax=4.5,
                ymin=-0.2,
                ymax=1.2,
                xlabel={$x$},
                ylabel={$y$},
                xlabel style={below right},
                ylabel style={above left},
                height=4cm,
                width=8cm,
                ytick={0,1},
                xtick={0,1,2,3,4},
                ]
    
                \addplot[red] plot (\x, { exp(-\x)) });
            \end{axis}
        \end{tikzpicture}
        \captionof{figure}{Exponential function $exp(x) = e^{-x}$.}
        \label{img:math:exp}
    \end{minipage}        
    \hfill
    \begin{minipage}{0.47\linewidth}
        \begin{tikzpicture}
            \begin{axis}[
                axis lines=center,
                samples=50,
                xmin=-0.5,
                xmax=4.5,
                ymin=-0.2,
                ymax=1.2,
                xlabel={$x$},
                ylabel={$y$},
                xlabel style={below right},
                ylabel style={above left},
                height=4cm,
                width=8cm,
                ytick={0,1},
                xtick={0,1,2,3,4},
                ]
    
                \addplot[red] plot (\x, { 1 - exp(-\x)) });
            \end{axis}
        \end{tikzpicture}
        \captionof{figure}{Inverted exponential function $exp'(x) = 1 - e^{-x}$.}
        \label{img:math:exp1}       
    \end{minipage}
\end{figure}

\noindent
When inverting $exp(x)$, the function $exp'(x)$ returns a value that can be directly used for the transparency of the cloud. The denser it gets, the more opaque it will be.

\clearpage
\subsubsection{Improving Noise}


\subsection{Light Transmittance and Light Scattering}


\begin{figure}[H]
    \centering
    \begin{tikzpicture}[scale=1.2]
        \tikzset{edge/.style = {-{Latex[length=3mm]},shorten >= -4pt}}
        \tikzset{shortedge/.style = {-{Latex[length=3mm]},shorten <=-4pt,shorten >= -4pt}}
        \tikzset{lightedge1/.style = {-{Latex[length=3mm]},shorten <=-4pt,shorten >= 1cm}}
        \tikzset{lightedge2/.style = {-{Latex[length=3mm]},shorten <=-4pt,shorten >= 1cm}}
        \tikzset{lightedge3/.style = {-{Latex[length=3mm]},shorten <=-4pt,shorten >= 1cm}}
        \tikzset{icon/.style = {font=\Large}}

        % icons
        \node[icon,rotate=35,anchor=west] (cam) at (0, 0) {\faVideoCamera};
        \node[icon] (light) at (1, 6) {\faLightbulbO};
        \node at (1, 6.5) {light source};

        % clouds
        \node (cloud) at (4.5, 3.5) {};
        \node[cloud, cloud puffs=15.7, cloud ignores aspect, minimum width=5cm, minimum height=3.5cm, align=center, draw] (cloud) at (cloud) {};

        % outer point
        \node (pOut) at (7, 5.25) {};

        % rays
        \draw[red, edge] (cam) -- (pOut) node[midway,above,sloped,xshift=-3cm] {};
        
        % cloud points
        \node[red] (p1) at (3.5, 2.625) {\textbullet};
        \node[red] (p2) at (4.5, 3.375) {\textbullet};
        \node[red] (p3) at (5.5, 4.125) {\textbullet};
        \node[red] (p4) at (4.5, 3.375) {\textbullet};
        \node[red] (p5) at (5.5, 4.125) {\textbullet};

        % light march rays
        \draw[cyan, lightedge1] (p1) -- (light) node[midway,above,sloped] {};
        \draw[cyan, lightedge2] (p2) -- (light) node[midway,above,sloped] {};
        \draw[cyan, lightedge3] (p3) -- (light) node[midway,above,sloped] {$R_{light}$};

        % light sample points
        \node[cyan] (l1) at (3.25, 2.95) {\textbullet};
        \node[cyan] (l2) at (4.15, 3.625) {\textbullet};
        \node[cyan] (l3) at (5.0, 4.325) {\textbullet};
        
        \node[cyan] (l5) at (3.0, 3.3) {\textbullet};
        \node[cyan] (l6) at (3.75, 3.95) {\textbullet};
        \node[cyan] (l7) at (4.5, 4.55) {\textbullet};
        
        \node[cyan] (l8) at (2.75, 3.65) {\textbullet};
        \node[cyan] (l9) at (3.35, 4.22) {\textbullet};
        \node[cyan] (l10) at (3.95, 4.775) {\textbullet};

        \end{tikzpicture}
    \caption{Sunlight transmittance sampling (1).}
    \label{img:tikz:prototypes:lightmarching1}
\end{figure}

\begin{figure}[H]
    \centering
    \begin{tikzpicture}[scale=1.2]
        \tikzset{edge/.style = {-{Latex[length=3mm]},shorten >= -4pt}}
        \tikzset{shortedge/.style = {-{Latex[length=3mm]},shorten <=-4pt,shorten >= -4pt}}
        \tikzset{lightedge1/.style = {-{Latex[length=3mm]},shorten <=-4pt,shorten >= 1cm}}
        \tikzset{lightedge2/.style = {-{Latex[length=3mm]},shorten <=-4pt,shorten >= 1cm}}
        \tikzset{lightedge3/.style = {-{Latex[length=3mm]},shorten <=-4pt,shorten >= 1cm}}
        \tikzset{icon/.style = {font=\Large}}

        % icons
        \node[icon,rotate=35,anchor=west] (cam) at (0, 0) {\faVideoCamera};
        \node[icon] (light) at (1, 6) {\faLightbulbO};
        \node at (1, 6.5) {light source};

        % clouds
        \node (cloud) at (4.5, 3.5) {};
        \node[cloud, cloud puffs=15.7, cloud ignores aspect, minimum width=5cm, minimum height=3.5cm, align=center, draw] (cloud) at (cloud) {};

        % outer point
        \node (pOut) at (7, 5.25) {};

        % rays
        \draw[red, edge] (cam) -- (pOut) node[midway,above,sloped,xshift=-3cm] {};
        
        % cloud points
        \node[red] (p1) at (3.5, 2.625) {\textbullet};
        \node[red] (p2) at (4.5, 3.375) {\textbullet};
        \node[red] (p3) at (5.5, 4.125) {\textbullet};
        \node[red] (p4) at (4.5, 3.375) {\textbullet};
        \node[red] (p5) at (5.5, 4.125) {\textbullet};

        % light march rays
        \draw[cyan, lightedge1] (p1) -- (light) node[midway,above,sloped] {};
        \draw[cyan, lightedge2] (p2) -- (light) node[midway,above,sloped] {};
        \draw[cyan, lightedge3] (p3) -- (light) node[midway,above,sloped] {$R_{light}$};

        % light sample points
        \node[cyan] (l1) at (3.25, 2.95) {\textbullet};
        \node[cyan] (l2) at (4.15, 3.625) {\textbullet};
        \node[cyan] (l3) at (5.0, 4.325) {\textbullet};
        
        \node[cyan] (l5) at (3.0, 3.3) {\textbullet};
        \node[cyan] (l6) at (3.75, 3.95) {\textbullet};
        \node[cyan] (l7) at (4.5, 4.55) {\textbullet};
        
        \node[cyan] (l8) at (2.75, 3.65) {\textbullet};
        \node[cyan] (l9) at (3.35, 4.22) {\textbullet};
        \node[cyan] (l10) at (3.95, 4.775) {\textbullet};

        \end{tikzpicture}
    \caption{Sunlight transmittance sampling  (2).}
    \label{img:tikz:prototypes:lightmarching2}
\end{figure}