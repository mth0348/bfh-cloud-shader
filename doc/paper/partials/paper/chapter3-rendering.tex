\section{Rendering techniques}

\subsection{Volumetric rendering}
\label{section:volumetric-rendering}

\subsubsection{Definition}
\Gls{volumetricrendering} describes a technique for generating a visual representation of data that is stored in a 3D volume. 
This especically comes to use for visual effects that are volumetric in nature, like fluids, clouds, fire, smoke, fog and dust, which all are extremely difficult or even impossible to model with geometric primitives.
\\
In addition to rendering such effects, volumetric rendering has become essential to scientific applications like medical imaging, for which a typical 3D data volume is a set of 2D slice images acquired by a CT or MRI scanner.
\emptyline
The data volume is also called a \textit{scalar field}, which associates a scalar value, or \textit{voxel}, to every point in the defined space.
This can be imagined like a 3D grid, where each point holds a single number. This number could, for example, represent the density of a cloud at that very point.
A voxel may also consist of more than just a single value, but rather a set, like an RGB color.

\subsubsection{Ray Marching with constant step}
To actually render the volume data, a method called \textit{\gls{raymarching}} is used. With it, the surface distance of the volumetric data is approximated by creating a ray from the camera to the object for each fragment processed in the fragment shader. The ray is then extended into the volume of the object and stepped forward until the surface is reached.

\begin{figure}[H]
    \centering
    \begin{tikzpicture}[scale=0.7]
        \draw[-{Latex[length=2mm]},thick,cyan] (2.5,3) node[below right]{stepped forward} -- (4.75,4.125);
        \node[cyan] at (3.0,3.25) {\textbullet};
        \node[cyan] at (3.5,3.50) {\textbullet};
        \node[cyan] at (4.0,3.73) {\textbullet};
        
        \draw (0,0) -- (5,0) -- (5,5) -- (0, 5) -- (0, 0);
        \draw (2,1) -- (7,1) -- (7,6) -- (2, 6) -- (2, 1);
        \draw (0,0) -- (2,1); \draw (5,0) -- (7,1); \draw (5,5) -- (7,6); \draw (0,5) -- (2,6); 

        \draw[thick,red] (-1.5,1) node[below right]{camera ray} -- (2.5,3);

    \end{tikzpicture}
    \caption{A ray is created from the camera to the object. From there, it is extended into the volume.}
\end{figure}

\pagebreak
\noindent
In \gls{raymarching}, the algorithm only knows when it has reached the surface, or to be precise when it is inside the actual object volume.
\\
With this information, it is only possible to extend the ray in steps of a predefined length until the inside of the object is reached.
With a constant step, the approximation of the surface distance is exactly as precise as the size of the constant step.
\\
Once the ray is inside the actual volume, the functions returns the distance for this ray.

\begin{figure}[H]
    \centering
    \begin{tikzpicture}[scale=1.2]
        \tikzset{edge/.style = {-{Latex[length=3mm]}}}
        \tikzset{smalledge/.style = {-{Latex[length=1mm]}}}

        % ray
        \draw[edge] (0,2) node[above,sloped,xshift=1cm]{ray} -- (10,3);
        \node (1) at (2,2.2) {\textbullet};
        \node (2) at (3,2.3) {\textbullet};
        \node (3) at (4,2.4) {\textbullet};
        \node (4) at (5,2.5) {\textbullet};
        \node (5) at (6,2.6) {\textbullet};
        \node (6) at (7,2.7) {\textbullet};
        \node[cyan] (7) at (8,2.8) {\textbullet};

        % surf dist
        \draw[red] (2,1.5) -- (2.5,2) -- (4,1.8) -- (5,1.5) -- (7.4,2) -- (7.7,4) -- (9,4.3);

        % ray steps
        \draw[smalledge] (1) edge[bend left=60] node [left]{} (2);
        \draw[smalledge] (2) edge[bend left=60] node [left]{} (3);
        \draw[smalledge] (3) edge[bend left=60] node [left]{} (4);
        \draw[smalledge] (4) edge[bend left=60] node [left]{} (5);
        \draw[smalledge] (5) edge[bend left=60] node [left]{} (6);
        \draw[smalledge] (6) edge[bend left=60] node [left]{} (7);

        % returns text
        \draw[cyan, shorten <=-0.2cm] (7) -- (9,3.7) -- (11,3.7) node[xshift=-1cm,above]{function returns};
        
        \end{tikzpicture}
    \caption{The function returns the evaluated surface distance, the precision of which being directly dependent on the step size.}
\end{figure}


\subsubsection{Traditional Ray Marching}
It is obvious to see that, for a constant step ray march to result in an accurate approximation of the surface distance, the step size is required to be relatively small.
This has a direct impact on performance and thus, is not a viable solution for the problem.
\\
In traditional \gls{raymarching}, an optimization for that has been developed. The algorithm does not blindly step forward, but instead tries to get as close to the real distance as possible.
After the volume is reached, the step size is decreased and the ray steps out of the volume again. It then tries to approximate the surface distance by stepping back and forward repeatedly in continuously smaller steps, thus converging towards the exact intersection.
Once the step size falls below a certain threshold, the distance approximation is assumed to be precise enough and the value is returned for that ray march.

\begin{figure}[H]
    \centering
    \begin{tikzpicture}[scale=1.2]
        \tikzset{edge/.style = {-{Latex[length=3mm]}}}
        \tikzset{smalledge/.style = {-{Latex[length=2mm]}}}

        % ray
        \draw[edge] (0,2) node[above,sloped,xshift=1cm]{ray} -- (10,3);
        \node (1) at (2,2.2) {\textbullet};
        \node (2) at (4,2.4) {\textbullet};
        \node (3) at (6,2.6) {\textbullet};
        \node[cyan] (4) at (8,2.8) {\textbullet};

        % surf dist
        \draw[red] (2,1.5) -- (2.5,2) -- (4,1.8) -- (5,1.5) -- (7.4,2) -- (7.7,4) -- (9,4.3);

        % reverse nodes
        \node[cyan] (5) at (7,2.7) {\textbullet};
        \node[cyan] (6) at (7.5,2.75) {\textbullet};

        % ray steps
        \draw[smalledge] (1) edge[bend left=60] node [left]{} (2);
        \draw[smalledge] (2) edge[bend left=60] node [left]{} (3);
        \draw[smalledge] (3) edge[bend left=60] node [left]{} (4);

        % rey reverse steps
        \draw[smalledge,cyan,shorten >=-0.1cm,shorten <=-0.1cm] (4) edge[bend left=60] node [left]{} (5);
        \draw[smalledge,cyan,shorten >=-0.1cm,shorten <=-0.1cm] (5) edge[bend left=60] node [left]{} (6);

        % close enough
        \draw[cyan, shorten <=-0.2cm] (6) -- (9,3.7) -- (11,3.7) node[xshift=-1cm,above]{step size small enough};
        
        \end{tikzpicture}
    \caption{The function returns the distance for this ray, which is the amount of steps times the step size.}
\end{figure}