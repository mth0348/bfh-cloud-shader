\section{Common Algorithms}

\subsection{Noise Generation}
\label{section:noise-generation}
Nature's unpredictability and plays a big role in the diversity and appearance of cloudscapes. In shaders, an approach to that \textit{randomness} is used called \textit{\gls{noise}}.
In order to be able to implement random \gls{noise}, several important topics need to be looked into. It is best to start with randomness in computer science and how it is handled inside a shader program.

\subsubsection{Random Numbers}
As expected, there is no magic function which returns a pure random number inside the seemingly predictable and rigid code environment.
So the question arises how to generate randomness, or at least an approximation to it.
\begin{figure}[H]
    \centering
    \begin{tikzpicture}[scale=1.0]
        \draw[very thin,color=lightgrey,step=1.046cm] (-0.1,-1.1) grid (10,1.5);
        \draw[->] (-1,0) -- (10,0) node[right] {$x$};
        \draw[->] (0,-1) -- (0,1.5) node[above] {$y$};
        \draw[red] plot[domain=0:3*pi, samples=720] (\x,{ sin(\x r) - floor(sin(\x r)) });
    \end{tikzpicture}
    \caption{Stuff.}
\end{figure}
