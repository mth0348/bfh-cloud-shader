\section{Project Management}
\label{section:projectmanagement}

\subsection{Schedule Adherence}
The following chart shows the original schedule (in grey) with a side-by-side comparison with the actual time spent for each task (in blue). It indicates that the schedule was met throughout the project.
\vspace{\baselineskip}

\begin{ganttchart}[
    vgrid={dotted},
    hgrid={draw=black!50, dotted},
    bar/.append style={fill=lightgray},
    x unit=0.65cm,
    ms/.append style={milestone/.append style={fill=gray, scale = 0.7, yshift=0.2cm}},
    ms2/.append style={milestone/.append style={fill=cyan, scale = 0.7, yshift=-0.2cm}},
    progress label text={},
    bar1/.append style={bar height=0.2},
    bar2/.append style={bar height=0.2},
    bar2/.append style={bar top shift=0.018cm},
    ]{1}{16}
    \gantttitle{Work weeks}{16} \\
    \gantttitlelist{1,...,16}{1} \\
    \ganttbar[bar1]{Organization}{1}{4}
    \ganttbar[bar2, bar/.append style={fill=cyan}]{}{1}{4} \\
    \ganttmilestone[ms]{Specification finished}{4}
    \ganttmilestone[ms2]{}{4} \\
    \ganttgroup{Documentation}{5}{15} \\
    \ganttbar[bar1]{Research}{5}{11}
    \ganttbar[bar2, bar/.append style={fill=cyan}]{}{5}{13} \\
    \ganttbar[bar1]{Prototyping}{7}{15}
    \ganttbar[bar2, bar/.append style={fill=cyan}]{}{7}{15} \\
    \ganttmilestone[ms]{Prototype 1 finished}{9}
    \ganttmilestone[ms2]{}{9} \\
    \ganttmilestone[ms]{Prototype 2 finished}{11}
    \ganttmilestone[ms2]{}{15} \\
    \ganttmilestone[ms]{Prototype 3 finished}{15} \\
    \ganttbar[bar1]{Finalizing}{16}{16}
    \ganttbar[bar2, bar/.append style={fill=cyan}]{}{16}{16}
\end{ganttchart}

\vspace{\baselineskip}
\noindent
As explained in \sectionref{section:projectmanagement:goals}, "prototype 3" was removed from the schedule, which is why the blue milestone is missing. Due to the unexpectedly freed up development time, the milestone for "prototype 2" was moved to the end of the segment.
\\
It is noteworthy that two more weeks were put into research. This is mainly because new methods and algorithms have been continuously researched during prototyping, which resulted in constant documentation of those findings.
It is not regarded as failed planning due to the fact that the extra time gained from the removed prototype allowed for it to happen harmlessly.
\\
Still, the total amount of invested time was about ten percent more than the originally estimated time budget of 128 hours. This is probably due to the fact that there was quite some effort put into the final prototype.

\clearpage
\subsection{Goal Discrepancies}
\label{section:projectmanagement:goals}
Originally, the following three prototypes were planned.
\begin{itemize}
    \item \gls{volumetricrendering}
    \item \gls{procedural} \gls{noise} generation
    \item \gls{raymarching}
\end{itemize}
During research and prototyping, it came clear that "\gls{raymarching}" is in fact a substantial part of \gls{volumetricrendering} instead of a completely different topic.
Hence, only two of the three listed prototypes were implemented. This change led to a significant boost in available development time for the other two prototypes, for which the results could now be fleshed out to a greater extent.

\subsection{Future Work}
The final prototype leaves a lot to work with for future projects, such as rendering of more cloud genera like the infamous, high-towering cumulonimbus clouds, performance optimizations and much more.

\subsubsection{Complete Weather System}
Another wonderful idea is to expand the shader into a fully-fledged weather system. Instead of having all the those technical \gls{parameters}, it would instead be dependent on temperature, moistness, altitude, highs and lows, weather fronts and many other meteorological, real variables.
It would automatically start to rain when the the conditions are met and the cloud shader movement would adjust itself by the wind strength and direction \gls{parameters}.

\subsubsection{Extensive Lighting Features}
Despite the considerable effort already put into lighting and illumination methods, there are still some features missing. One of those are \textit{god rays}, the volumetric light shafts that shine through gaps in clouds, giving the scene even more depth.
Other absent features are the sun's and moon's halo: A bright circle around the celestial body.

\subsubsection{Measure Realism}
As described in \sectionref{section:prototypes:realismcheck}, there are several possible approaches to measure the realism of the clouds. This opens up another potential future project.