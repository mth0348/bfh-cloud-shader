\section{Project management}

\subsection{Schedule}
The time frame of the semester spans over exactly 16 weeks. Being worth 4 ECTS points, this project assumes a maximum work load of 8 hours per week, resulting in a total of 128 hours. 
\vspace{\baselineskip}
\\
Over the course of the term, the project will be split into four primary task groups, namely organization, research, prototyping and finalization.
Put into relation with the duration of the project, the estimated schedule looks like this:
\vspace{\baselineskip}

\begin{ganttchart}[
    vgrid={dotted},
    hgrid={draw=black!50, dotted},
    bar/.append style={fill=lightgray},
    x unit=0.65cm,
    milestone node/.append style={fill=orange}
    ]{1}{16}
    \gantttitle{Work weeks}{16} \\
    \gantttitlelist{1,...,16}{1} \\
    \ganttbar{Organization}{1}{4} \\
    \ganttmilestone{Specification finished}{4} \\
    \ganttgroup{Documentation}{5}{15} \\
    \ganttbar{Research}{5}{11} \\
    \ganttbar{Prototyping}{7}{15} \\
    \ganttmilestone{Prototype 1 finished}{9} \\
    \ganttmilestone{Prototype 2 finished}{11} \\
    \ganttmilestone{Prototype 3 finished}{15} \\
    \ganttbar{Finalizing}{16}{16}
\end{ganttchart}

\vspace{\baselineskip}
\begin{flushleft}
The milestones \emph{Prototype finished} for prototypes 1, 2, and 3 are referring to the prototypes about \gls{volumetric}, \gls{noise} and \gls{raymarching}, respectively.
\\
\vspace{\baselineskip}
During research and prototyping, the documentation will be continuously updated.
\\
For each task group, the following distribution of time and effort is estimated:
\newline
\newline
\begin{tabular}{|c|c|}
    \hline
    \textbf{Task group}  & \textbf{Predicted effort}\\ \hline
    Organization        & 10\%                      \\ \hline
    Research            & 35\%                      \\ \hline
    Prototyping         & 50\%                      \\ \hline
    Finalizing          & 5\%                       \\ \hline
\end{tabular}
\end{flushleft} 

%==============================================================

\clearpage
\subsection{Project Organization}
A meeting will be held bi-weekly to discuss the progress of the project, possibly arisen issues as well as planned work for the upcoming two weeks.
\vspace{\baselineskip}
\newline
\noindent Mandatory participants are:
\vspace{\baselineskip}
\newline
\noindent\begin{tabular}{|l|l|}
    \hline
    \textbf{Name}       & \textbf{Role}         \\ \hline
    Matthias Thomann    & Author                \\ \hline
    Prof. Urs Künzler   & Tutor and reviewer    \\ \hline
\end{tabular}

\vspace{\baselineskip}
\noindent Should a physical meeting be impossible for some reason, an online meeting via Microsoft Teams will be held instead.

%==============================================================

\subsection{Project Results}

\subsubsection{Documentation}
The following documents must be submitted for grading:
\begin{itemize}
    \item Requirement specification
    \item Project documentation indcluding:
    \begin{itemize}
        \item Conceptual formulation
        \item Comparison and details of methods and algorithms
        \item Details of implementation
        \item Result report
    \end{itemize}
\end{itemize}

\subsubsection{Implementation of the Shader}
The Unity project, including the implemented shader code, will be managed and stored in the given Gitlab repository\cite{gitlab}. This will also serve as a form of submission for grading.

\subsubsection{Presentation}
A presentation will be held on the second last friday of the term, June 5th, 2020.

\subsubsection{Submission terms}
The project must be submitted in digital form on the last friday of the term, June 12th, 2020 before 12pm.