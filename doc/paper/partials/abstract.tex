\section*{Abstract}
Clouds contribute a great deal to the overall ambience in games and can be the cherry on top by filling the sky with life.
To get as close as possible to real clouds, this project engages in researching and prototyping a \gls{procedural}, volumetric cloud shader.
\\
In order to achieve \gls{volumetricrendering}, the document dives into the concept of \gls{raymarching}, a group of methods used to render a 3D data set inside a container box to make it appear volumetric.
Several variants of it are expanded on, like constant step, traditional, and sphere-traced \gls{raymarching}. Additionally, to account for perception of depth, the volume can be shaded with the aid of \gls{surfacenormal} estimation.
\\
In the second part, 2D and 3D \gls{noise} generation algorithms like Perlin's and the Voronoi algorithm are explained in detail. With \gls{fbm}, the different layers of noise are then merged into one highly detailed noise texture.
\\
At last, the goal of the project was to create prototypes in Unity Editor displaying both volumetric rendering and noise algorithms, of which all were created successfully.
Armed with the combined knowledge of the research results and prototypes, a final shader was created, able to render a completely \gls{procedural} and volumetric cloudscape.
\\
For future work, the shader could be expanded into a fully-fledged weather system with meteorologically accurate formation of clouds, rain, snow and much more.